\documentclass[french]{article}
  \usepackage{babel}
  \usepackage[utf8]{inputenc}   % LaTeX
  \usepackage[T1]{fontenc}      % LaTeX
  \usepackage[colorlinks]{hyperref}
  \usepackage{lipsum}
  \usepackage{hyperref}
	\usepackage{graphicx}
	\usepackage{tabularx}
  	\graphicspath{{./fig/}}
  \usepackage{desclist}
  \setlength{\parindent}{0pt} %retrait indentation début para
  
\addto\extrasfrench{ 
  \def\itemautorefname{situation}  
}  
  

  \frenchbsetup{og=«,fg=»}      % caractères « et » sont les guillemets
  % \frenchbsetup{ItemLabeli=}
  
  \title{\vspace{2cm}
  \LARGE
  \textbf{Utilisation des calculatrices}
  \normalsize
  \vspace{6cm}}
  \author{David Beauchemin \\ 
  \vspace{2cm} Alexandre Parent \vspace{2cm}}
  
%%% macro pour ajouter auteur par chapitre
\makeatletter
\newcommand{\chapterauthor}[1]{%
  {\parindent0pt\vspace*{-15pt}%
  \linespread{1.1}\large\scshape#1%
  \par\nobreak\vspace*{35pt}}
  \@afterheading%
}
\makeatother

% déclaration de formule pour annuité
\DeclareRobustCommand{\annuity}[1]{%
\def\arraystretch{0}%
\setlength\arraycolsep{.7pt}%
\setlength\arrayrulewidth{.3pt}% 
\begin{array}[b]{@{}c|}\hline
\\[\arraycolsep]%
\scriptstyle #1%
\end{array}%
}

% exemple d'utilisation de la fonction annuity
% \ddot{a}_{\annuity{n}i}

\begin{document}

\maketitle

\newpage
\input{section/sec_license}
\newpage

%%%%%%%%%%% Abstract %%%%%%%%%%%%%%%%%%%
\begin{abstract}
Ce guide s'adresse en premier lieu aux étudiants qui débutent dans le baccalauréat en actuariat. Il s'agit d'un guide aidant les nouveaux utilisateurs des calculatrices dans l'utilisation des éléments clés des calculatrices. Les auteurs croient que l'utilisation intelligente des différentes fonctionnalités des calculatrices contribue à faciliter le calcul dans certains examens. Le temps étant souvent un facteur crucial dans les examens, les raccourcis présentés ici contribuent au succès des étudiants dans les examens professionnel et du baccalauréat.
{\tiny} 
\end{abstract}

\newpage
\tableofcontents

\setlength{\parskip}{1.8ex}


\pagebreak

%%%%%%%%%%%%% Content %%%%%%%%%%%%%%%%%%%%%%%%%%%%%%
\section{TI-30XS Multiview}
\chapterauthor{Alexandre Parent}

\section*{Mise en contexte}
Salut à tous! J'écris cette section dans le but d'aider les nouveaux utilisateurs de la calculatrice TI-30XS Multiview. En gros, j'ai réalisé que même à ma deuxième année, j'apprenais encore des fonctionnalités que j'aurais aimé connaître plus tôt. On se sert de notre calculatrice presque tous les jours, ce serait bien de savoir tout ce qu'elle est capable de faire! Il n'y a rien de trop prétentieux à ce document en passant, je veux simplement partager mes conseils à ceux qui les veulent. Si vous avez des questions, des commentaires, des ajouts ou des corrections à apporter à ce document, s'il vous plaît, écrivez-moi. En passant, j'ai voulu faire ce document pour avoir quelque chose de personnalisé aux étudiants en actuariat de l'Université Laval et pour apprendre à travailler avec LaTeX, mais toutes ces informations sont aussi disponibles sur Youtube sans avoir à chercher trop loin.

\subsection{Fonctions à connaître dès le début}
\label{sec:fonctiondebut}

Je ne prendrai pas le temps de décrire chaque touche de la calculatrice, tout le monde sait déjà comment fonctionne une calculatrice scientifique. Donc, voici les quelques fonctions que je ne connaissais pas au début, mais que j'aime bien.

\subsubsection{Décimal à fraction}
Une touche toute simple que j'ai connue trop tard: La touche \raisebox{-1mm}{\includegraphics[height=5mm]{B_FlechesOpposees}} %avec deux flèches opposées
au-dessus de la touche \texttt{enter} (en bas à droite). Elle sert à convertir le format d'une réponse entre nombre à virgule et fraction.

\subsubsection{Les fonctions de probabilité}
Le bouton \raisebox{-1mm}{\includegraphics[height=5mm]{B_Prb}} permet de faire des combinaisons, des permutations et des factoriels. Rien de compliqué, il faut simplement l'essayer (pour les combinaisons ${n \choose x}$, on doit mettre \texttt{n}, puis pèser sur \raisebox{-1mm}{\includegraphics[height=5mm]{B_Prb}}, puis \texttt{2}, et finalement \texttt{x}. Si tu ne sais pas encore ce qu'est une combinaison ou un factoriel, tu vas le savoir bien assez vite.

\subsubsection{Garder une valeur en mémoire}
Je pense que la seule fois que j'ai utilisé cette fonction, c'était avec la fonction \texttt{Table} décrite plus loin afin de changer plus facilement le paramètre d'une loi (fonction de survie de la Gompertz dans un examen papier trop long en vie I). La fonction peut aussi servir lors des examens professionnels quand on est habitué d'avoir plein de numéros semblables avec les mêmes variables, pour stocker un taux d'intérêt par exemple.

Pour garder une valeur en mémoire, on entre cette valeur (ou elle sera déjà entrée parce que c'est le résultat d'un calcul), puis on appuie sur \raisebox{-1mm}{\includegraphics[height=5mm]{B_Sto}} pour dire qu'on veut la sauvegarder. Finalement, on appuie sur \raisebox{-1mm}{\includegraphics[height=5mm]{B_Xyztabc}} pour choisir quelle lettre représentera la valeur. Pour voir le contenu de chaque variable ou la réutiliser, on pèse sur \raisebox{-1mm}{\includegraphics[height=5mm]{B_2nd}} et \raisebox{-1mm}{\includegraphics[height=5mm]{B_Sto}}. Pour la réutiliser plus rapidement, on pèse sur \raisebox{-1mm}{\includegraphics[height=5mm]{B_Xyztabc}} jusqu'à ce que ça sorte la bonne variable.

Pour revenir à mon exemple, c'est que la formule à entrer était trop longue pour être mise dans la fonction \texttt{Table} et j'avais à calculer pour plusieurs valeurs d'un paramètre. J'ai donc pu faire un \texttt{Recall} de \texttt{x} dans ma formule dans \texttt{Table} et changer la valeur de \texttt{x} avec cette méthode.

\subsubsection{La fonction \texttt{Table}}
Cette fonction sert à essayer plusieurs valeurs de \texttt{x} dans une fonction mathématique. Par exemple, je m'en suis servi en Compléments de mathématiques pour trouver les racines d'une fonction ou dans les examens professionnels pour essayer les différents choix de réponse. 

Pour s'en servir, on pèse sur \raisebox{-1mm}{\includegraphics[height=5mm]{B_Table}} et on entre la fonction de \texttt{x}, le bouton \raisebox{-1mm}{\includegraphics[height=5mm]{B_Xyztabc}} sert à mettre les \texttt{x}. Ensuite, après avoir pesé sur \texttt{enter}, il y aura différentes options pour décider des valeurs de x à entrer, rien de trop compliqué.

\subsubsection{La fonction \texttt{Data}}
La fonction \texttt{Data} est celle que je me sers le plus souvent, mais il faut prendre le temps d'apprendre à bien l'utiliser. C'est dur d'expliquer clairement dans quels cas elle peut servir, donc je vais me contenter de l'expliquer.

On appuie sur \raisebox{-1mm}{\includegraphics[height=5mm]{B_Data}} pour voir les trois colonnes de la fonction \texttt{Data}. Dans chacune des colonnes (on utilise \raisebox{-2.5mm}{\includegraphics[height=7mm]{B_4Fleches}} pour sélectionner les différentes cellules), on peut entrer des nombres ou des formules qui dépendent des autres colonnes.

On appuie encore sur \raisebox{-1mm}{\includegraphics[height=5mm]{B_Data}} pour accéder au menu de \texttt{Data}. Ici, on peut effacer le contenu d'une colonne ou de toutes en descendant dans les options. On peut aussi utiliser le nombre à gauche pour sélectionner une option (on appuie sur \texttt{1} pour effacer le contenu de la première colonne). Sinon, en haut de l'écran, à droite de \texttt{CLEAR}, il y a \texttt{FORMULA}. En y allant et en sélectionnant la première option (\texttt{Add/Edit Frmla}), on peut entrer une formule dans la colonne où on était. 

On peut maintenant entrer la formule qui peut être fonction des deux autres colonnes (mais seulement sur une même ligne. Par exemple, on ne peut pas entrer une formule en fonction de la cellule au-dessus). Pour mettre la valeur d'une autre colonne dans la formule, on appuie une troisième fois sur \raisebox{-1mm}{\includegraphics[height=5mm]{B_Data}} et on choisit la bonne colonne.

\subsection{Fonctions statistiques}

On entre dans une section qui ne sera pas vraiment utile pour les étudiants en première session. Je pense que les fonctions statistiques de la calculatrice vont surtout servir pour les cours de statistiques (2e session), IARD I, Modèles linéaires (3e session) et IARD II (4e session), en plus des examens MFE et C.

\subsubsection{Entrer des données}
\label{subsec:EntrerDonnees}

Pour entrer les données, on utilise la fonction \texttt{data}, déjà présentée à la \autoref{sec:fonctiondebut}. De mémoire, il y a trois façons de recevoir des données dans une question:

\begin{enumerate}

\item Chaque donnée est différente, ex: (10, 12, 15, 20)
\label{itm:donneedifferente}

\item Les données sont répétées, ex: (2, 2, 2, 4, 4, 7, 7, 7, 7, 7, 7, 7)
\label{itm:donneerepetee}

\item On a une variable explicative \texttt{x} et une variable réponse \texttt{y}, donc chaque observation a deux données associées
\label{itm:donneeassociee}

\end{enumerate}

La \autoref{itm:donneedifferente} est intuitive. On entre les données dans une colonne, tout simplement.

Pour la \autoref{itm:donneerepetee}, on entre les valeurs dans une colonne et le nombre de répétitions de chaque valeur dans une autre colonne.

La \autoref{itm:donneeassociee} est liée au cours de modèles linéaires. Les fonctions statistiques à deux variables seront présentées dans la sous-section \nameref{subsec:statdeuxvariables}.

\subsubsection{Utiliser \texttt{Stat} avec une variable}

Une fois que les données sont entrées dans \texttt{data}, il ne reste qu'à appuyer sur \raisebox{-1mm}{\includegraphics[height=5mm]{B_2nd}} et \raisebox{-1mm}{\includegraphics[height=5mm]{B_Data}}, puis à sélectionner \texttt{1: 1-Var Stats}. On choisit ensuite comment les données étaient entrées, donc dans quelle colonne étaient les données et, pour la \autoref{itm:donneerepetee} (où les données sont répétées), on choisit dans \texttt{FRQ:} la colonne qui donne le nombre de fois que chaque donnée est répétée.

C'est tout! On a maintenant accès à toutes ces statistiques:

\begin{desclist}{\sf}{ \rm\hfill}[minX]

\item [n] nombre d'observations
\item [x] moyenne empirique
\item [Sx] variance empirique non biaisée (divisée par ${n - 1}$)
\item [$\sigma_{x}$] variance empirique biaisée (divisée par ${n}$)
\item [$\sum x$] somme des observations
\item [$\sum x^{2}$] somme des observations au carré
\item [minX] minimum des observations
\item [\ldots]

\end{desclist}



\subsubsection[Utiliser \texttt{Stat} avec deux variables]{Utiliser \texttt{Stat} avec deux variables (modèles linéaires)}
\label{subsec:statdeuxvariables}

Pour utiliser les fonctions de statistiques à deux variables, liées à un modèle de régression linéaire simple que l'on voit dans le cours de modèles linéaires à la 3e session, il faut d'abord entrer les différentes observations comme à la \ref{subsec:EntrerDonnees}. J'utiliserai la notation $\hat{Y_i} = \beta_0 + \beta_1 \cdot X_i$, donc $X$ est la variable explicative et $Y$ la variable réponse.

Donc, une fois les $X$ entrés dans une colonne et les $Y$ dans une autre (dans \texttt{data}), on pèse sur \raisebox{-1mm}{\includegraphics[height=5mm]{B_2nd}} et \raisebox{-1mm}{\includegraphics[height=5mm]{B_Data}}. Ensuite, on choisit \texttt{2: 2-Var Stats} et on indique à la calculatrice quelle colonne est associée à quelle variable.

Terminé! Je veux juste faire remarquer les statistiques qui peuvent être très pratiques:

\begin{desclist}{\sf}{ \rm\hfill}[minX]

\item [a] Valeur de $\beta_1$
\item [b] Valeur de $\beta_0$
\item [r] Mis au carré, ça nous donne $R^2$
\item [x'] Donne la valeur de $X_i$ qui serait associée à la valeur de $Y_i$ qu'on entre après avoir sélectionné cette fonction
\item [y'] Même principe, on choisit cette fonction puis on entre une valeur de $X_i$, ça sort la valeur prédite de $Y_i$

\end{desclist}


\newpage
\section*{BA II +}
\chapterauthor{David Beauchemin}

\section*{Mise en contexte}
Cette section est dédiée à la calculatrice financière. Cette calculatrice est utile seulement pour le cours de mathématique financière et l'examen FM. Il n'y a aucun avantage à utiliser la calculatrice dans d'autre cours, la présentation étant très limitée. Je présente ici les fonctions principalement utilisées pour l'examen FM. Le guide a été construit lorsque j'ai fait mon examen FM en 2016, certains éléments pourraient changer.
\begin{enumerate}
\item Boverman offre aussi un \emph{PDF} d'explication, le document est disponible sur le Google Drive du groupe d'échange de document dans la section du cours de \href{https://drive.google.com/open?id=0B6kXivc6X9LITmdBVFVWSDAxeE0}{mathématique financière}.
\item J'ai mis aussi des résumés de formule du cours précédent et de la \emph{SOA} qui sont disponible sur mon \href{https://rawgit.com/davebulaval/Study_material_act/master/Résumé\%20FM/ResumeFM_ASM10e.pdf}{\emph{GitHub}}.
\item \textbf{Notes sur la légende d'écriture} : 
Un symbole + signifie prochaine touche à \textit{cliquer} est la suivante.

\end{enumerate}

\subsection{Format d'affichage, valeur future, valeur actualisée et taux nominaux}

\textit{*Chapitre 1 dans le livre}

\subsubsection{Format d'affichage}
\fbox{2ND} + \fbox{format} + nombre de décimale + \fbox{enter}

\subsubsection{Valeur future}
\begin{enumerate}

\item Accumulation simple
\\ (1 + taux d'intérêt) + \fbox{$ y^{x} $} + valeur de l'exposant (x) + \fbox{X} + montant à accumulé + \fbox{=}
\item Fonction TVM
\\ Légende :

\begin{enumerate}
\item \fbox{N} périodes ;
\item \fbox{I$ / $Y} taux d'intérêt par période ;
\item \fbox{PV} Valeur présente ;
\item \fbox{PMT} Paiement (annuité) ;
\item \fbox{FV} Valeur accumulée ;
\item Astuce : La fréquence du taux d'intérêt peut-être modifié. On pourrait mettre le taux annuel effectif et jouer avec les paramètres de la calculatrice pour avoir un taux d'intérêt mensuel.
\\ Voici comment, \fbox{I$ / $Y} et régler à 12 pour avoir un mensuel. De base, pour ne pas faire d'erreur laisser à 1. Mais toujours utile de savoir cette fonction.
\end{enumerate}

\item \textbf{Comment utilisé TVM :}
\\ Nombre périodes + \fbox{N} + taux d'intérêt + \fbox{I$ / $Y} + valeur à accumulé + \fbox{+$ / $-} + \fbox{PV} + \fbox{CPT} + \fbox{FV}
\\ 
\item \textbf{Astuce :} Pour afficher la valeur d'un des paramètres utilisée dans TVM, \fbox{RCL} + \fbox{N} ou \fbox{PV}... 
\\
\item \textbf{Astuce :}
\begin{LARGE}
Ne pas oublier de \textit{clear} les valeurs!!
\end{LARGE}
\\ \fbox{2ND} + \fbox{CLR TVM}

\end{enumerate}

\subsubsection{Trouver le taux d'intérêt}
Nombre de périodes + \fbox{N} + montant à accumuler + \fbox{PV} + montant futur + \fbox{FV} + \fbox{CPT} + \fbox{I$ / $Y }

\subsubsection{Trouver le nombre de période}
Taux d'intérêt + \fbox{I$ / $Y } + valeur présente + \fbox{+$ / $- } + \fbox{PV} + montant future + \fbox{FV} + \fbox{CPT} + \fbox{N}

\subsubsection{Taux nominal et TVM}
Comme les taux nominaux sont divisés par le nombre de périodes, on peut simplement faire : \\
Nombre périodes + \fbox{N} + ($i^{(m)} \div m$) + \fbox{=} + \fbox{I$ / $Y} + valeur à accumulé + \fbox{+$ / $-} + \fbox{PV} + \fbox{CPT} + \fbox{FV}
\\

\subsubsection{Taux équivalent}

Convertir un taux nominal en effectif : 

\fbox{2ND} + \fbox{ICONV} + \emph{NOM} (taux nominal) + \fbox{ENTER} + \fbox{$\Downarrow$} jusqu'à \emph{$C / Y$} (nombre de périodes) + \fbox{ENTER} + \fbox{$\Uparrow$} jusqu'à \emph{EFF} + \fbox{CPT}
\\
\\ Pour trouver un taux nominal on \fbox{CPT} \emph{NOM} et on fixe le taux effectif dans \fbox{EFF}.
\\
\\ Pour trouver un taux d'escompte, convertir \emph{d} en \emph{i}.

\subsection{Annuité et calculatrice}
\label{Annuité et calculatrice}

Pour l'utilisation de TVM, voir section \ref{sec:accumulation}, la majorité des notions de cette section sont identique pour les annuités.

\subsubsection{Annuité due et \textit{Begin}}
\label{Begin}

La calculatrice possède une fonction \textit{Begin} qui permet de calculer l'annuité avec un paiement en début de période sans manipulation algébrique. Par contre, il faut la remettre à \textit{End} pour revenir à une annuité immédiate. Voici comment faire ;
\fbox{2ND} + \fbox{BGN} + \fbox{2ND} + \fbox{SET}.
Refaire la même procédure pour revenir à \textit{End}.

\subsubsection{Annuité à progression arithmétique}
\label{annuité et TI-30XS}
Voici une astuce pour calculer à partir des formules de la section \ref{Sub:Chap4:increasing} et \ref{Sub:Chap4:decreasing}. On utilise la calculatrice TI-30XS multiview, afin de ne pas se mélanger dans l'équation on utilise la touche \fbox{$\frac{n}{d}$}.

\subsection{Amortissement}
\label{Amortissement}

\subsubsection{TVM et fonction d'amortissement}
\label{ann:chap:amortissement}
Tout d'abord, on enregistre les informations dans la fonction TVM (\textit{N}, \textit{$I/Y$}, \textit{PV}, \textit{$-$PMT}). Par la suite, \fbox{2ND} + \fbox{AMORT} + (P1) = paiement désiré + \fbox{ENTER} + \fbox{$\downarrow$} + (P2) = paiement désiré (le même) + \fbox{ENTER}, par la suite avec les flèches on peut voir le principal, la balance et l'intérêt payé. Si on veut changer de paiement, on retourne à P1 et P2 pour modifier l'information\footnote{ASM p 313-314}. 
\\Note : P1 indique la ligne de début et P2 indique la ligne de fin. Donc, si P1 = 1 et P2= 3, il va s'agir du capital et de l'intérêt payé entre 1 et 3.

\subsection{Obligations}
\label{obligation}

Deux méthodes d'approche pour résoudre le prix des obligations avec la calculatrice :

\subsubsection{TVM et obligation}
\label{anex:TVM et obligation}

La méthode TVM :  nombre de coupons + \fbox{N} + taux d'intérêt + \fbox{I/Y} + montant du coupon + \fbox{PMT} + Valeur de rachat + \fbox{FV} + \fbox{CPT} + \fbox{PV}

\subsubsection{\textit{Bond worksheet}}
\label{anex:bond worksheet}

La meilleure source pour cette section est le livre d'instruction de la calculatrice voir le lien \href{https://drive.google.com/open?id=0B6kXivc6X9LIMXI4T1QtcTFQNzA}{page 65}. 


\end{document}
