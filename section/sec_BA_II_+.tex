\begin{enumerate}
\item Boverman offre aussi un \emph{PDF} d'explication, le document est disponible sur le Google Drive du groupe d'échange de document dans la section du cours de \href{https://drive.google.com/open?id=0B6kXivc6X9LITmdBVFVWSDAxeE0}{mathématique financière}.
\item J'ai mis aussi des résumés de formule du cours précédent et de la \emph{SOA} qui sont disponible sur mon \href{https://rawgit.com/davebulaval/Study_material_act/master/Résumé\%20FM/ResumeFM_ASM10e.pdf}{\emph{GitHub}}.
\item \textbf{Notes sur la légende d'écriture} : 
Un symbole + signifie prochaine touche à \textit{cliquer} est la suivante.

\end{enumerate}

\subsection{Format d'affichage, valeur future, valeur actualisée et taux nominaux}

\textit{*Chapitre 1 dans le livre}

\subsubsection{Format d'affichage}
\fbox{2ND} + \fbox{format} + nombre de décimale + \fbox{enter}

\subsubsection{Valeur future}
\begin{enumerate}

\item Accumulation simple
\\ (1 + taux d'intérêt) + \fbox{$ y^{x} $} + valeur de l'exposant (x) + \fbox{X} + montant à accumulé + \fbox{=}
\item Fonction TVM
\\ Légende :

\begin{enumerate}
\item \fbox{N} périodes ;
\item \fbox{I$ / $Y} taux d'intérêt par période ;
\item \fbox{PV} Valeur présente ;
\item \fbox{PMT} Paiement (annuité) ;
\item \fbox{FV} Valeur accumulée ;
\item Astuce : La fréquence du taux d'intérêt peut-être modifié. On pourrait mettre le taux annuel effectif et jouer avec les paramètres de la calculatrice pour avoir un taux d'intérêt mensuel.
\\ Voici comment, \fbox{I$ / $Y} et régler à 12 pour avoir un mensuel. De base, pour ne pas faire d'erreur laisser à 1. Mais toujours utile de savoir cette fonction.
\end{enumerate}

\item \textbf{Comment utilisé TVM :}
\\ Nombre périodes + \fbox{N} + taux d'intérêt + \fbox{I$ / $Y} + valeur à accumulé + \fbox{+$ / $-} + \fbox{PV} + \fbox{CPT} + \fbox{FV}
\\ 
\item \textbf{Astuce :} Pour afficher la valeur d'un des paramètres utilisée dans TVM, \fbox{RCL} + \fbox{N} ou \fbox{PV}... 
\\
\item \textbf{Astuce :}
\begin{LARGE}
Ne pas oublier de \textit{clear} les valeurs!!
\end{LARGE}
\\ \fbox{2ND} + \fbox{CLR TVM}

\end{enumerate}

\subsubsection{Trouver le taux d'intérêt}
Nombre de périodes + \fbox{N} + montant à accumuler + \fbox{PV} + montant futur + \fbox{FV} + \fbox{CPT} + \fbox{I$ / $Y }

\subsubsection{Trouver le nombre de période}
Taux d'intérêt + \fbox{I$ / $Y } + valeur présente + \fbox{+$ / $- } + \fbox{PV} + montant future + \fbox{FV} + \fbox{CPT} + \fbox{N}

\subsubsection{Taux nominal et TVM}
Comme les taux nominaux sont divisés par le nombre de périodes, on peut simplement faire : \\
Nombre périodes + \fbox{N} + ($i^{(m)} \div m$) + \fbox{=} + \fbox{I$ / $Y} + valeur à accumulé + \fbox{+$ / $-} + \fbox{PV} + \fbox{CPT} + \fbox{FV}
\\

\subsubsection{Taux équivalent}

Convertir un taux nominal en effectif : 

\fbox{2ND} + \fbox{ICONV} + \emph{NOM} (taux nominal) + \fbox{ENTER} + \fbox{$\Downarrow$} jusqu'à \emph{$C / Y$} (nombre de périodes) + \fbox{ENTER} + \fbox{$\Uparrow$} jusqu'à \emph{EFF} + \fbox{CPT}
\\
\\ Pour trouver un taux nominal on \fbox{CPT} \emph{NOM} et on fixe le taux effectif dans \fbox{EFF}.
\\
\\ Pour trouver un taux d'escompte, convertir \emph{d} en \emph{i}.

\subsection{Annuité et calculatrice}
\label{Annuité et calculatrice}

Pour l'utilisation de TVM, voir section \ref{sec:accumulation}, la majorité des notions de cette section sont identique pour les annuités.

\subsubsection{Annuité due et \textit{Begin}}
\label{Begin}

La calculatrice possède une fonction \textit{Begin} qui permet de calculer l'annuité avec un paiement en début de période sans manipulation algébrique. Par contre, il faut la remettre à \textit{End} pour revenir à une annuité immédiate. Voici comment faire ;
\fbox{2ND} + \fbox{BGN} + \fbox{2ND} + \fbox{SET}.
Refaire la même procédure pour revenir à \textit{End}.

\subsubsection{Annuité à progression arithmétique}
\label{annuité et TI-30XS}
Voici une astuce pour calculer à partir des formules de la section \ref{Sub:Chap4:increasing} et \ref{Sub:Chap4:decreasing}. On utilise la calculatrice TI-30XS multiview, afin de ne pas se mélanger dans l'équation on utilise la touche \fbox{$\frac{n}{d}$}.

\subsection{Amortissement}
\label{Amortissement}

\subsubsection{TVM et fonction d'amortissement}
\label{ann:chap:amortissement}
Tout d'abord, on enregistre les informations dans la fonction TVM (\textit{N}, \textit{$I/Y$}, \textit{PV}, \textit{$-$PMT}). Par la suite, \fbox{2ND} + \fbox{AMORT} + (P1) = paiement désiré + \fbox{ENTER} + \fbox{$\downarrow$} + (P2) = paiement désiré (le même) + \fbox{ENTER}, par la suite avec les flèches on peut voir le principal, la balance et l'intérêt payé. Si on veut changer de paiement, on retourne à P1 et P2 pour modifier l'information\footnote{ASM p 313-314}. 
\\Note : P1 indique la ligne de début et P2 indique la ligne de fin. Donc, si P1 = 1 et P2= 3, il va s'agir du capital et de l'intérêt payé entre 1 et 3.

\subsection{Obligations}
\label{obligation}

Deux méthodes d'approche pour résoudre le prix des obligations avec la calculatrice :

\subsubsection{TVM et obligation}
\label{anex:TVM et obligation}

La méthode TVM :  nombre de coupons + \fbox{N} + taux d'intérêt + \fbox{I/Y} + montant du coupon + \fbox{PMT} + Valeur de rachat + \fbox{FV} + \fbox{CPT} + \fbox{PV}

\subsubsection{\textit{Bond worksheet}}
\label{anex:bond worksheet}

La meilleure source pour cette section est le livre d'instruction de la calculatrice voir le lien \href{https://drive.google.com/open?id=0B6kXivc6X9LIMXI4T1QtcTFQNzA}{page 65}. 
